\chapter{Sejarah dan Perkembangan Kecerdasan Buatan}
Buku umum teori lengkap yang digunakan memiliki Semakin maju nya teknologi membuat banyak sekali perangkat pintar yang kemudian mengadopsi teknologi Kecerdasan buatan atau biasa disebut dengan Artificial Intelligence (AI). Dengan adanya kehadiran AI ini menimbulkan banyak sekali manfaat tentunya, ia dapat meringankan beban pekerjaan masuia serta membuatnya lebih efektif dan juga efisien. Kecerdasan buatan dapat disimpulkan sebagai suatu kecerdasan atau keahlian yang pada dasarnya merupakan buatan manusia. Yang mana kecerdasan otak manusia itu ialah alami dimiliki dan tumbuh sepanjang seseorang tersebut bernafas. Setelah itu, seseorang dengan kecerdasan alami ini lah kemudian mulai merangkai sebuah sistem atau perangkat, yang bertujuan supaya perangkat ini dapat memudahkan suatu pekerjaan sendiri maupun orang lain dalam tanda kutip manusia. Untuk alat yang berhasil diciptakan dan teknologi yang berhasil dilahirkan inilah yang kemudian disebut sebagai AI.
Pada awalnya ai mulai dikenal publik itu karena ia memiliki kemampuan baik dalam menitu kegiatan manusia. Yang kemudian memunculkan banyak anggapan negetif, dimana manusia akan digeser oleh mesin. Akan tetapi semakin berkembangnya waktu, anggapan itupun berubah menjadi anggapan positif, karena dengan adanya AI ini dapat mempermudah dan mempuat pekerjaan menjadi mudah, cepat dan juga efektif efisien. Dengan adanya AI ini  dinilai dapat membantu pekerjaan sehari-hari dan juga mudah untuk dikendalikan. Yang kemudian AI sudah dijadikan sebagai teman dan bukan lagi dianggap sebagai musuh atau pun ancaman bagi manusia.


Adapun manfaat dari teknologi AI ini antara lain sebagai berikut :
\begin{enumerate}
	\item
	      Membantu meminimalkan kesalahan
	\item
	      Solusi untuk hemat energi
	\item
	      Berperan dalam eksplorasi kekayaan alam
	\item
	      Hemat SDM
	\item
	      Bermanfaat di bidang kesehatan
\end{enumerate}

\section{Supervised Learning}
Supervised Learning adalah tugas pengumpulan data untuk menyimpulkan fungsi dari data pelatihan berlabel. Data pelatihan terdiri dari serangkaian contoh pelatihan. Dalam supervised learning, setiap contoh adalah pasangan yang terdiri dari objek input (biasanya vektor) dan nilai output yang diinginkan(juga disebut sinyal pengawasan super). Algoritma pembelajaran yang diawasi menganalisis data pelatihan dan menghasilkan fungsi yang disimpulkan, yang dapat digunakan untuk memetakan contoh-contoh baru.Supervised Learning menyediakan algoritma pembelajaran dengan jumlah yang diketahui untuk mendukung penilaian dimasa depan. Chatbots, mobil self-driving, program pengenalan wajah, sistem pakar dan robot adalah beberapa sistem yang dapat menggunakan pembelajaran yang diawasi atau tidak. Model Supervised Learning memiliki beberapa keunggulan dibandingkan pendekatan tanpa pengawasan, tetapi mereka juga memiliki keterbatasan. Sistem lebih cenderung membuat penilaian bahwa manusia dapat berhubungan, misalnya karena manusia telah memberikan dasar untuk keputusan. Namun, dalam kasus metode berbasis pengambilan, Supervised Learning mengalami kesulitan dalam menangani informaasi baru. Jika suatu sistem dengan kategori untuk mobil dan truk disajikan dengan sepeda, misalnya ia harus salah dikelompokkan dalam satu kategori ata yang lain. Namun. jika sistem AI bersifat generatif, ia mungkin tidak tahu apa sepeda itu tetapi akan dapat mengenalinya sebagai milik kategori yang terpisah

\section{Klasifikasi dan Regresi}
Klasifikasi yaitu pendekatan pembelajaran yang diawasi dimana program komputer belajar dari input data yag diberikan  kepadanya dan kemudian menggunakan pembelajaran ini untuk mengklarifikasikan pengamatan baru. Regresi adalah membehas mengenai masalah ketika variable output adalah nilai rill atau berkelanjutan contohnya seperti "gaji" atau "berat". banyak model yang berbeda dapat digunakan makan, yang paling sederhana adalah regresi linier. ia mencoba untuk menyesuaikan data degan hyper-plane terbaik yang melewati poin.

\section{Unsupervised Learing}
Unsupervised Learning berbeda dengan Supervised Leraning. Perbedaannya ialah unsupervised learning tidak memiliki data latih, sehingga dari data yang ada kita mengelompokan data tersebut menjadi 2 ataupun 3 bagian dan seterusnya. Unsupervised Learning adalah pelatihan algoritma kecerdasan buatan (AI) menggunakan informasi yang tidak diklasifikasikan atau diberi label dan memungkinkan algoritma untuk bertindak atas informasi tersebut tanpa bimbingan.
Dalam Unsupervised Learning, sistem AI dapat mengelompokkan informasi yang tidak disortir berdasarkan persamaan dan perbedaan meskipun tidak ada kategori yang disediakan

\section{Data set, Traning set, Testing set}
Dataset adalah objek yang merepresentasikan data dan juga relasi yang ada di memory. Strukturnya mirip dengan data di database, namun bedanya dataset berisi koleksi dari data table dan data relation. Training Set adalah set digunakan oleh algoritma klassifikasi . Dapat dicontohkan dengan : decision tree, bayesian, neural network dll. Testing Set adalah set yang digunakan untuk mengukur sejauh mana classifier berhasil melakukan klasifikasi dengan benar. Ini berfungsi sebagai meterai persetujuan, dan Anda tidak menggunakannya sampai akhir.